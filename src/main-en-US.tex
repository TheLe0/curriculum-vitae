\documentclass[10pt, letterpaper]{article}

% Packages:
\usepackage[
    ignoreheadfoot, % set margins without considering header and footer
    top=2 cm, % seperation between body and page edge from the top
    bottom=2 cm, % seperation between body and page edge from the bottom
    left=2 cm, % seperation between body and page edge from the left
    right=2 cm, % seperation between body and page edge from the right
    footskip=1.0 cm, % seperation between body and footer
    % showframe % for debugging 
]{geometry} % for adjusting page geometry
\usepackage{titlesec} % for customizing section titles
\usepackage{tabularx} % for making tables with fixed width columns
\usepackage{array} % tabularx requires this
\usepackage[dvipsnames]{xcolor} % for coloring text
\definecolor{primaryColor}{RGB}{0, 79, 144} % define primary color
\usepackage{enumitem} % for customizing lists
\usepackage{fontawesome5} % for using icons
\usepackage{amsmath} % for math
\usepackage[
    pdftitle={Tosin's CV},
    pdfauthor={Leonardo Tosin},
    pdfcreator={LaTeX with RenderCV},
    colorlinks=true,
    urlcolor=primaryColor
]{hyperref} % for links, metadata and bookmarks
\usepackage[pscoord]{eso-pic} % for floating text on the page
\usepackage{calc} % for calculating lengths
\usepackage{bookmark} % for bookmarks
\usepackage{lastpage} % for getting the total number of pages
\usepackage{changepage} % for one column entries (adjustwidth environment)
\usepackage{paracol} % for two and three column entries
\usepackage{ifthen} % for conditional statements
\usepackage{needspace} % for avoiding page brake right after the section title
\usepackage{iftex} % check if engine is pdflatex, xetex or luatex

% Ensure that generate pdf is machine readable/ATS parsable:
\ifPDFTeX
    \input{glyphtounicode}
    \pdfgentounicode=1
    % \usepackage[T1]{fontenc} % this breaks sb2nov
    \usepackage[utf8]{inputenc}
    \usepackage{lmodern}
\fi



% Some settings:
\AtBeginEnvironment{adjustwidth}{\partopsep0pt} % remove space before adjustwidth environment
\pagestyle{empty} % no header or footer
\setcounter{secnumdepth}{0} % no section numbering
\setlength{\parindent}{0pt} % no indentation
\setlength{\topskip}{0pt} % no top skip
\setlength{\columnsep}{0cm} % set column seperation
\makeatletter
\let\ps@customFooterStyle\ps@plain % Copy the plain style to customFooterStyle
\patchcmd{\ps@customFooterStyle}{\thepage}{
    \color{gray}\textit{\small Leonardo Tosin - Page \thepage{} of \pageref*{LastPage}}
}{}{} % replace number by desired string
\makeatother
\pagestyle{customFooterStyle}

\titleformat{\section}{\needspace{4\baselineskip}\bfseries\large}{}{0pt}{}[\vspace{1pt}\titlerule]

\titlespacing{\section}{
    % left space:
    -1pt
}{
    % top space:
    0.3 cm
}{
    % bottom space:
    0.2 cm
} % section title spacing

\renewcommand\labelitemi{$\circ$} % custom bullet points
\newenvironment{highlights}{
    \begin{itemize}[
        topsep=0.10 cm,
        parsep=0.10 cm,
        partopsep=0pt,
        itemsep=0pt,
        leftmargin=0.4 cm + 10pt
    ]
}{
    \end{itemize}
} % new environment for highlights

\newenvironment{highlightsforbulletentries}{
    \begin{itemize}[
        topsep=0.10 cm,
        parsep=0.10 cm,
        partopsep=0pt,
        itemsep=0pt,
        leftmargin=10pt
    ]
}{
    \end{itemize}
} % new environment for highlights for bullet entries


\newenvironment{onecolentry}{
    \begin{adjustwidth}{
        0.2 cm + 0.00001 cm
    }{
        0.2 cm + 0.00001 cm
    }
}{
    \end{adjustwidth}
} % new environment for one column entries

\newenvironment{twocolentry}[2][]{
    \onecolentry
    \def\secondColumn{#2}
    \setcolumnwidth{\fill, 4.5 cm}
    \begin{paracol}{2}
}{
    \switchcolumn \raggedleft \secondColumn
    \end{paracol}
    \endonecolentry
} % new environment for two column entries

\newenvironment{header}{
    \setlength{\topsep}{0pt}\par\kern\topsep\centering\linespread{1.5}
}{
    \par\kern\topsep
} % new environment for the header

\newcommand{\placelastupdatedtext}{% \placetextbox{<horizontal pos>}{<vertical pos>}{<stuff>}
  \AddToShipoutPictureFG*{% Add <stuff> to current page foreground
    \put(
        \LenToUnit{\paperwidth-2 cm-0.2 cm+0.05cm},
        \LenToUnit{\paperheight-1.0 cm}
    ){\vtop{{\null}\makebox[0pt][c]{
        \small\color{gray}\textit{Last updated in September 2024}\hspace{\widthof{Last updated in September 2024}}
    }}}%
  }%
}%

% save the original href command in a new command:
\let\hrefWithoutArrow\href

% new command for external links:
\renewcommand{\href}[2]{\hrefWithoutArrow{#1}{\ifthenelse{\equal{#2}{}}{ }{#2 }\raisebox{.15ex}{\footnotesize \faExternalLink*}}}


\begin{document}
    \newcommand{\AND}{\unskip
        \cleaders\copy\ANDbox\hskip\wd\ANDbox
        \ignorespaces
    }
    \newsavebox\ANDbox
    \sbox\ANDbox{}

    \placelastupdatedtext
    \begin{header}
        \textbf{\fontsize{24 pt}{24 pt}\selectfont Leonardo Tosin}

        \vspace{0.3 cm}

        \normalsize
        \mbox{{\color{black}\footnotesize\faMapMarker*}\hspace*{0.13cm}Garibaldi, RS - Brazil}%
        \kern 0.25 cm%
        \AND%
        \kern 0.25 cm%
        \mbox{\hrefWithoutArrow{leob.tosin@hotmail.com}{\color{black}{\footnotesize\faEnvelope[regular]}\hspace*{0.13cm}leob.tosin@hotmail.com}}%
        \kern 0.25 cm%
        \AND%
        \kern 0.25 cm%
        \mbox{\hrefWithoutArrow{tel:+55 (54) 99929-9478}{\color{black}{\footnotesize\faPhone*}\hspace*{0.13cm}+55 (54) 99929-9478}}%
        \kern 0.25 cm%
        \AND%
        \kern 0.25 cm%
        \mbox{\hrefWithoutArrow{https://leotosin.com.br/}{\color{black}{\footnotesize\faLink}\hspace*{0.13cm}leotosin.com.br}}%
        \kern 0.25 cm%
        \AND%
        \kern 0.25 cm%
        \mbox{\hrefWithoutArrow{https://www.linkedin.com/in/leonardo-tosin-b57406112/}{\color{black}{\footnotesize\faLinkedinIn}\hspace*{0.13cm}leonardo-tosin}}%
        \kern 0.25 cm%
        \AND%
        \kern 0.25 cm%
        \mbox{\hrefWithoutArrow{https://github.com/TheLe0}{\color{black}{\footnotesize\faGithub}\hspace*{0.13cm}TheLe0}}%
    \end{header}

    \vspace{0.3 cm - 0.3 cm}


    \section{About}



        
        \begin{onecolentry}
            I am a software developer with over 7 years of experience, specializing in building applications from the ground up to go-live. Throughout my career, I have worked across various sectors, including enterprise, financial, and retail applications. I have a strong focus on high-performance projects, particularly in areas such as search engines, databases, and other related technologies, ensuring that solutions are robust and efficient while aligning client needs with best development practices.
            
            I'm a computer science bachelor, built as a final paper a gamification plataform for touristics itineraries. You can here more about in my \href{https://www.youtube.com/watch?v=xZLdsME5gGU}{video} (it's in portuguese language).
        \end{onecolentry}

        \vspace{0.2 cm}
    
    \section{Links}

    \section{Education}


        \begin{twocolentry}{
            
            
        \textit{March 2013 – Aug 2022}}
            \textbf{Universidade de Caxias do Sul}

            \textit{BS in Computer Science}
        \end{twocolentry}

        \vspace{0.10 cm}
        \begin{onecolentry}
            \begin{highlights}
                \item The final grade was 9.8/10
                \item I built as the final paper a mobile application called Itinehapp, a.k.a Your happy itinerary app, that tourists can build their owns itineraries for destinations to all over the world and share with other people.
            \end{highlights}
        \end{onecolentry}



    
    \section{Experience}

        \begin{twocolentry}{
        \textit{Remote}    
            
        \textit{Jan 2024 - currently}}
            \textbf{Senior Full Stack Engineer}
            
            \textit{CI\&T}
        \end{twocolentry}

        \vspace{0.10 cm}
        \begin{onecolentry}
            \begin{highlights}
                \item Working with a US-based client on a project to develop an ECM system capable of uploading millions of documents daily, featuring a user-friendly and fast search mechanism based on metadata classification..
                \item Involved in backend development to implement system logic and business rules, frontend development to design user-friendly and intuitive interfaces, and DevOps/infrastructure to create Infrastructure as Code (IaC) solutions that automate the entire platform's build and deployment processes
                \item Improved search queries and mechanisms, enabling sub-second query responses for millions of data points.
                \item Implemented a Disaster Recovery strategy to prevent system outages.
                \item Developed a platform for seamless data migration between legacy systems and the new one.
                \item Tools Used: C\#, .NET, .NET Framework, MongoDb, AWS, Apache Lucene, VPC, AWS Lambda, API Gateway, AWS S3, Vector Search and QuickSight.
            \end{highlights}
        \end{onecolentry}

        \begin{twocolentry}{
        \textit{Remote}    
            
        \textit{Jul 2022 - Jan 2024}}
            \textbf{Software Engineer}
            
            \textit{Warren Investimentos}
        \end{twocolentry}

        \vspace{0.10 cm}
        \begin{onecolentry}
            \begin{highlights}
                \item Survey of technical and functional requirements for new futures with other member of the team, for building scalable and resilient new features for the business.
                \item  Helped new people to join the team and understand our system and business with coding reviews, 1v1 and pair programming.
                \item Development new features mainly with .NET and NodeJS with highly test coverage (unit, integration, e2e and stress).
                \item Integration with frontends by BFFs, crossing informations with other microservices.
                \item Created some web front-ends applications for the backoffice daily operations.
                \item Troubleshooting of bugs and war rooms to solve problems and incidents.
                \item Scrum cerimonies like daily, plannings, theoretical refinement, technical refinament and sprints retrospectives.
                \item Documentation of the features and applications flows
                \item Cloud resources monitoring and managements, both on the cloud plataform portal and by IaC (Terraform).
                \item Direct contact with the product's stakeholders to better understanding of their day-by-day routine, needs, frustrations and visions for a better user friendly features implementations.
                \item Tools Used: C\#, .NET, .NET Framework, MySQL, Apache Kafka, VueJS, NodeJS, MongoDb, AWS, ElasticSearch, DataDog, Open Telemetry and ECS.
            \end{highlights}
        \end{onecolentry}

        \begin{twocolentry}{
        \textit{Carlos Barbosa, RS}    
            
        \textit{Sep 2019 – Jul 2022}}
            \textbf{Software Developer}
            
            \textit{Cooperativa Santa Clara}
        \end{twocolentry}

        \vspace{0.10 cm}
        \begin{onecolentry}
            \begin{highlights}
                \item Fixed and improved the retail system, responsible to import the orders from the POS to the ERP for create and invoice the orders.
                \item Implemented a batch finacial settlement using concurrency programming, in some minutes can settle hundreds of securities, when before the user needed to do this manually.
                \item Created a static files storages on the cloud, using Azure Blob Storage, to store images, PDFs and other text files. To do the integration between the systems to the cloud used the AzCopy tool provided by Microsoft, that you can map a directory on the machine that the Azure service keeps listen to them and when theirs some event (a new file, deleted some file or modified) is updated on the Azure.
                \item Created a serverless REST API using Azure Funcions and Azure API Management, to authenticate and authorize users on the Active Directory.
                \item Created the backend of the E-commerce plataform of the company, using REST, WSDL and SOAP to communicate and integrate with other systems (internals, payment gateway, carriers and others). I participated since the beginning of the project, not only programming, but modling the software and solution architecture.
                \item Tools Used: C\#, .NET, .NET Framework, SQL Server, Reporting Services, Azure, Azure Blob Storage, Azure Functions, NodeJS, ReactJS.
            \end{highlights}
        \end{onecolentry}
        
        
        \begin{twocolentry}{
        \textit{Garialdi, RS}    
            
        \textit{Apr 2017 – Sep 2019}}
            \textbf{Full Stack Developer}
            
            \textit{NuvolaHost Serviços de TI LTDA}
        \end{twocolentry}

        \vspace{0.10 cm}
        \begin{onecolentry}
            \begin{highlights}
                \item Modernizing web page layouts, focusing on producing responsive UIs.
                \item Built features and systems using TDD (Test-Driven Development) during the development process.
                \item Assisting in the implementation of DevOps, improving and automating the entire testing and deployment workflow for new software versions. Emphasizing running applications on Kubernetes and managing clusters through OpenShift.
                \item Migrated all company repositories from SVN to Git for version control.
                \item Built a Mail Marketing SaaS platform, that sens thousand of e-mails per day.
                \item Tools Used: PHP, Zend Framework, MySQL, JQuery, Ruby on Rails, Docker, JS, HTML and CSS.
            \end{highlights}
        \end{onecolentry}


        \vspace{0.2 cm}

        \begin{twocolentry}{
        \textit{Caxias do Sul, RS}    
            
        \textit{Jul 2014 – Mar 2015}}
            \textbf{Research Initiation}
            
            \textit{Microsoft}
        \end{twocolentry}

        \vspace{0.10 cm}
        \begin{onecolentry}
            \begin{highlights}
                \item I worked on a master's project called "Mantas Network", which focused on the use of polymer-based mats for oil, gasoline, and petroleum leak extraction in offshore environments.
                \item My main role was conducting scientific research on articles about new methods and techniques to be used. I performed practical tests and utilized statistical and probabilistic algorithms to calculate the effectiveness of the methods employed.
                \item Tools Used: Python, R, Matlab, Power BI and Excel.
            \end{highlights}
        \end{onecolentry}



    
    \section{Articles}

    \begin{onecolentry}
        \begin{highlightsforbulletentries}


        \item \href{https://medium.com/@TheLe0/web-scrapping-with-net-and-selenium-webdriver-d8a888756733}{Web Scrapping with .NET and Selenium Webdriver}
        \item \href{https://medium.com/@TheLe0/cloud-flavours-on-azure-the-different-types-of-cloud-services-6e8a12919d78}{Cloud flavours on Azure: The different types of cloud services}
        \item \href{https://medium.com/@TheLe0/learning-how-to-use-aws-cdk-with-net-as-the-iac-to-your-projects-c9806e5ea739}{Learning how to use AWS CDK with .NET as the IaC to your projects}
        \item \href{https://medium.com/@TheLe0/how-to-easily-build-and-deploy-a-website-on-azure-static-web-apps-32e62861a2e5}{How to easily build and deploy a website on Azure Static Web Apps}

        \end{highlightsforbulletentries}
    \end{onecolentry}

    
    \section{Projects}



        
        \begin{twocolentry}{
            
            
        \textit{\href{https://fun-with-flags.info/}{fun-with-flags.com}}}
            \textbf{Fun with Flags}
        \end{twocolentry}

        \vspace{0.10 cm}
        \begin{onecolentry}
            \begin{highlights}
                \item A website that offers a comprehensive list of all countries worldwide, along with their respective capitals. Each country card, when clicked, unveils intriguing facts about that nation
                \item Tools Used: C\#, .NET, Azure, CosmosDb, ChatGpt, Angular and Blob Storage
            \end{highlights}
        \end{onecolentry}


        \vspace{0.2 cm}

        \begin{twocolentry}{
            
            
        \textit{\href{https://todo-sphere.com/}{todo-sphere.com}}}
            \textbf{TODO Sphere}
        \end{twocolentry}

        \vspace{0.10 cm}
        \begin{onecolentry}
            \begin{highlights}
                \item A straightforward and responsive TODO app, developed in accordance with software development best practices.
                \item Tools Used: Angular, PHP, Laravel, Container Apps, Container Registry, Docker, Azure and MongoDb.
            \end{highlights}
        \end{onecolentry}


        \vspace{0.2 cm}

        \begin{twocolentry}{
            
            
        \textit{\href{https://www.berteleadvocacia.com.br/}{berteleadvocacia.com.br}}}
            \textbf{Bertele Advocacia}
        \end{twocolentry}

        \vspace{0.10 cm}
        \begin{onecolentry}
            \begin{highlights}
                \item A website for a law firm
                \item Tools Used: PHP, Blade, Laravel, Azure, Docker, Container Apps and Container Registry.
            \end{highlights}
        \end{onecolentry}



    
    \section{Technologies}



        
        \begin{onecolentry}
            \textbf{Languages:} C\#, SQL, JS/TS, Golang and Python
        \end{onecolentry}

        \vspace{0.2 cm}

        \begin{onecolentry}
            \textbf{Technologies:} .NET, Microsoft SQL Server, MongoDb, Azure, AWS, Angular, Docker, Apache Kafka, Terraform
        \end{onecolentry}


    

\end{document}
